\documentclass{article}
\usepackage{listings}
\usepackage{amsmath}
\usepackage{fullpage}
\usepackage{tabularx}
\usepackage{graphicx}
\usepackage{cite}
\begin{document}
\lstset{language=Java}
\title{CS317 homework 11}
\author{Yuchen Hou}
\maketitle

\section{RE}
\subsection{RE vs R}
\begin{align*}
L \in RE &\implies \exists TM \ M_1 \ accepting \ L \\
\overline{L} \in RE &\implies \exists TM \ M_2 \ accepting \ \overline{L}
\end{align*}
We can construct TM M to recognize L as follows:
\begin{enumerate}
  \item run $M_1, M_2$ in parallel on input w
  \item if $M_1 $ accepts w ($\iff w \in L$), accept w
  \item if $M_2$ accepts w($\iff w \in \overline{L} \iff w \notin L $),
  reject w
\end{enumerate}
TM M recognizes $L \implies L \in R$.
\subsection{RE closure property}
Proof by contradiction with assumption REs are closed under complement. Pick
$L \in RE \land L\notin R$. From the assumption, $\overline{L} \in
RE$. From the previous proof, $L \in RE \land \overline{L} \in RE \implies L
\in R$. But $L\notin R$. This is a contradiction. Thus REs are not closed
under complement.

\section{R closure property: complement}
$\forall L \in R \ \exists TM \ M$ recognizing L. We can construct TM M' to
recognize $\overline{L}$ as follows:
\begin{enumerate}
  \item run M on input w
  \item if M accepts w($\iff w \in L \iff w \notin \overline{L}$), reject w
  \item if M rejects w($\iff w \notin L \iff w \in \overline{L}$), accept w
\end{enumerate}
TM M' recognizes $\overline{L} \implies \overline{L} \in R$. Thus Rs are closed
under complement.

\section{Equivalent Turing machines}
$\forall$ TM M accepting L, we can construct a new TM M' accepting L, by adding
a new garbage transition to M: $\delta(q, \gamma) = (q, \gamma, S)$ where $q
\notin Q \land \gamma \in \Gamma$. By repeating this modification we can get
infinitely many different TMs accepting L. This means there are infinite many
equivalent programs to any program (e.g. programs that differ only in comments).

\section{R and decidable problem}
The problem is equivalent to L = \{M: TM M contains the same number of
L-instructions and R-instructions\}. From homework 8 problem 2 $\exists PDA \ M'
\ accepting \ L \implies L \in CFL \implies L \in R$. Thus the problem is
decidable.

\section{R and decidable problem}
The problem is equivalent to L = \{M: $\exists TM \ M'$ containing the same
number of L-instructions and R-instructions and L(M') = L(M)\}. $\exists \ TM \
N$ recognizing L: a TM accepting any input! Because $\forall M, \ M \in L$. We
can construct TM M' using the method to construct equivalent TM demonstrated in
problem 3: keep adding garbage L-instructions or R-instructions till M' contains
the same number of L-instructions and R-instructions. TM N recognizes $L
\implies L \in R$. Thus the problem is decidable.
\end{document}