\documentclass{article}
\usepackage{listings}
\usepackage{amsmath}
\usepackage{fullpage}
\usepackage{tabularx}
\usepackage{graphicx}
\usepackage{cite}
\begin{document}
\lstset{language=Java}
\title{CS317 homework 11}
\author{Yuchen Hou}
\maketitle

\section{RE}
\subsection{RE vs R}
We have the following observations:
\begin{align*}
L \in RE &\implies \exists TM \ M_1 : L = L(M_1) \\
\overline{L} \in RE &\implies \exists TM \ M_2 : \overline{L} = L(M_2)
\end{align*}
Now we construct TM M to recognize L:
\begin{enumerate}
  \item $\forall$ input w, M runs $M_1$ and $M_2$ in parallel on w
  \item if $M_1 $ accepts w($\iff w \in L$), M accepts w
  \item if $M_2$ accepts w($\iff w \in \overline{L} \iff w \notin L $), M
  rejects w
\end{enumerate}
M recognizes L $\implies L \in R$.
\subsection{RE closure property}
Proof by contradiction with assumption: REs are closed under complement. Pick
L, such that $L \in RE \land L\notin R$. From the assumption, $\overline{L} \in
RE$. From the previous proof, $L \in RE \land \overline{L} \in RE \implies L
\in R$, which contradicts $L\notin R$. Therefore, REs are not closed under
complement.

\section{R closure property: complement}
$\forall L \in R \ \exists TM \ M$ such that M recognizes L, which means:
\begin{enumerate}
  \item $\forall w \in L$, M accepts w
  \item $\forall w \notin L$, M rejects w
\end{enumerate}
Now we construct TM N to recognize $\overline{L}$:
\begin{enumerate}
  \item $\forall w$ N copies it to the tape of M
  \item N runs M
  \item if M accepts w($\iff w \in L \iff w \notin \overline{L}$), N rejects w
  \item if M rejects w($\iff w \notin L \iff w \in \overline{L}$), N accepts w
\end{enumerate}
As N accepts $\overline{L}$ and always halts, $\overline{L} \in R$. As $\forall
L \in R \ \overline{L} \in R$, Rs are closed under complement.

\section{Equivalent Turing machines}
For any TM M, we can get a new TM N such that L(M) = L(N), by modifying M this
way:
\begin{enumerate}
  \item create a state $q \notin Q$
  \item add transition $\delta(q_0, \gamma) = (q, \gamma, S)$
  \item add transition $\delta(q, \gamma) = (q_0, \gamma, S)$
\end{enumerate}
where $\gamma \in \Gamma$. Then by repeating this modification we can get
infinitely many TMs such that they all accept the same language. This means
there are infinite number of equivalent programs to any program (a trivial
example is programs that differ only in comments).

\section{R and decidable problem}
To prove the problem is decidable, it is equivalent to prove:
L = \{M: M is TM and M contains the same number of L-instructions and
R-instructions\} is R. Now we construct a counter machine N to recognize L:
\begin{enumerate}
  \item N has 2 counters: x and y
  \item N reads the encoding of M: for each R-instruction, increment x by 1; for
  each L-instruction, increment y by 1;
  \item after reading the encoding of M, N repeatedly decrements x and y by 1
  and test them against 0;
  \item when x is 0, if y is 0, N accepts input; if y is not 0, N rejects input;
\end{enumerate}
Therefore, N (and a equivalent TM) recognizes L. Then L is R and the problem is
decidable.

\section{R and decidable problem}
To prove the problem is decidable, it is equivalent to prove:
L = \{M: M is TM and there is a TM M' such that M' contains the same number of
L-instructions and R-instructions and L(M') = L(M)\} is R. Now we construct a TM
$M_0$ to recognize L:
\begin{enumerate}
  \item construct TM $M_1$ to recognize R $L_1$ = \{M: M is TM and L(M') =
  L(M)\}
  \item construct TM $M_2$ to recognize R $L_2$ = \{M: M is TM and M contains
  the same number of L-instructions and R-instructions\}
  \item From the closure property of R, $L_3 = L_1 \cap L_2 \in R$
  \item $\forall M$, if $L_3$ is not empty, $M_0$ accepts M, otherwise $M_0$
  rejects M
\end{enumerate}
Therefore $M_0$ recognizes L, and L is R and the problem is decidable.
\end{document}