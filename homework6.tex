\documentclass{article}
\usepackage{listings}
\usepackage{amsmath}
\usepackage{fullpage}
\usepackage{tabularx}
\usepackage{graphicx}
\usepackage{cite}
\begin{document}
\lstset{language=Java}
\title{CS317 homework 6}
\author{Yuchen Hou}
\maketitle
\section{Minimal finite automaton}
Figure Attached.
\section{Closure property: set operation}
\begin{lstlisting}
Boolean wordXExists(FiniteAutomaton M1, FiniteAutomaton M2) {
	RegularLanguage L1 = L(M1); // Kleene theorem
	RegularLanguage L2 = L(M2); // Kleene theorem
	RegularLanguage L = L1 - L2; // closure propertiy: set operation
	if (L.isEmpty()) {
		return false;
	} else {
		return true;
	}
}
void main() {
	// checking whether there is a word x such that M1 accepts x
	// but M2 does not accept x
	wordXExists(M1, M2);
	// checking whether there is a word x such that M2 accepts x
	// but M1 does not accept x
	wordXExists(M2, M1);
}
\end{lstlisting}
\section{Closure property: reverse}
From Kleene theorem, L is regular $\iff \exists$ finite automaton M =
($\Sigma, Q, q_0, \delta, F$) and M accepts L. We can convert M into another
finite automaton to accept $L^r$ with the following reverse method:
\begin{lstlisting}
class FiniteAutomaton M {
	Set<Char> Sigma; // set of characters
	Set<State> Q; // set of states
	State q0; // initial state
	Map<(State, State), Char> delta; // transition function
	Set<State> F; // set of final states

	void reverse() {
		State newAcceptingState = new State();
		for (State q: F) {
			delta.put((q, newAcceptingState), lambda);
		}
		F.clear();
		F.add(newAcceptingState);
		Q.add(newAcceptingState);
		State newInitialState = F.iterator().next();
		F.clear();
		F.add(q0);
		q0 = newInitialState;
		delta.reverseAllTransitionDirections();
	}
}
\end{lstlisting}
$\forall x \in L$, $M^r$ accepts $x^r$. Therefore, finite automaton $M^r$
accepts $L^r$. From Kleene theorem, $L^r$ is regular.
\section{Closure property: prefix}
From Kleene theorem, L is regular $\iff \exists$ finite automaton M =
($\Sigma, Q, q_0, \delta, F$) and M accepts L. We can convert M into another
finite automaton to accept Prefix(L) with the following prefix method:
\begin{lstlisting}
class FiniteAutomaton M {
	Set<Char> Sigma; // set of characters
	Set<State> Q; // set of states
	State q0; // initial state
	Map<(State, State), Char> delta; // transition function
	Set<State> F; // set of final states

	void prefix() {
		State newAcceptingState = new State();
		for (State q: Q) {
			if (q.hasPathToAnAcceptingState() {
				delta.put((q, newAcceptingState), lambda);
			}
		}
		F.add(newAcceptingState);
		Q.add(newAcceptingState);
	}
}
\end{lstlisting}
Finite automaton M can be converted into a new automaton to accept Prefix(L)
using the prefix method. From Kleene theorem, Prefix(L) is regular.
\section{Closure property: half}
From previous 2 answers, if L is regular, Prefix(L) and Prefix($L^r$) are
both regular, and there are finite automata $M_1$ and $M_2$ that accept them:
\begin{align*}
Prefix(L) =& L(M_1)\\
Prefix(L^r) =& L(M_2)\\
M_1 =& (\Sigma, Q, {q_0}_1, \delta_1, F_1)\\
M_2 =& (\Sigma, Q, {q_0}_2, \delta_2, F_2)
\end{align*}
For every transition $\delta(q_1, \sigma) = q_2$ in $M_2$, where
$\sigma \neq \lambda$, add transitions $\delta(q_1, \sigma_i) = q_2, \forall
\sigma_i \in \Sigma$ to $M_2$. Now $M_2$ is converted to a non-deterministic
finite automaton, we can convert it to a deterministic finite automaton $M_3$:
\begin{align*}
M_3 = (\Sigma, Q_3, {q_0}_3, \delta_3, F_3)
\end{align*}
Now we can construct a finite automaton $M_4 = (\Sigma,
Q_4, {q_0}_4, \delta_4, F_4)$ to accept the language half(L), where
\begin{align*}
\Sigma =& \Sigma\\
Q_4 =& QQ_3\\
{q_0}_4 =& {q_0}_1{q_0}_3\\
\delta_4(qq_3, \sigma) =& \delta_1(q, \sigma)\delta_3(q_3, \sigma)\\
F_4 =& \{qq_3 : \forall q \in Q, q_3 \in Q_3 s.t. q \in q_3 \}
\end{align*}
From Kleene theorem, half(L) is regular.
\end{document}