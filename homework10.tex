\documentclass{article}
\usepackage{listings}
\usepackage{amsmath}
\usepackage{fullpage}
\usepackage{tabularx}
\usepackage{graphicx}
\usepackage{cite}
\begin{document}
\lstset{language=Java}
\title{CS317 homework 10}
\author{Yuchen Hou}
\maketitle

\section{RE to TM}
\begin{align*}
\gamma &\in \{b, c\}\\
\delta(q_0, \gamma) &= (q_1, \gamma, R) \text{ // find 1 b or c}\\
\delta(q_1, \gamma) &= (q_1, \gamma, R) \text{ // skip all other b and c}\\
\delta(q_1, \#) &= (q_2, \#, R) \text{ // and find \#}\\
\delta(q_2, a) &= (q_3, a, R) \text{ // find 1 a}\\
\delta(q_3, a) &= (q_3, a, R) \text{ // skip all other a}\\
\delta(q_3, B) &= (q_f, B, S) \text{ // and reach the end, accept input}
\end{align*}

\section{RE to TM}
\begin{align*}
\gamma &= \{a, b\}\\
\gamma_1 &= \{a, b, \#\}\\
\delta(q_0, \gamma) &= (q_1, \gamma, S) \text{ // find 1 a or b}\\
&\text{ // begin: while (first symbol != \#)}\\
\delta(q_1, \gamma) &= (q_{\gamma,+}, B, R) \text{ // read the first symbol
$\gamma$, remember it, remove it from string}\\
\delta(q_{\gamma,+}, \gamma_1) &= (q_{\gamma,+}, \gamma_1, R) \text{ // skip all
symbols}\\
\delta(q_{\gamma,+}, B) &= (q_{\gamma,-}, B, L) \text{ // and reach the end}\\
\delta(q_{\gamma,-}, \gamma) &= (q_2, B, L) \text{ // verify the last symbol is
also $\gamma$, remove it from string}\\
\delta(q_2, \gamma_1) &= (q_2, \gamma_1, L) \text{ // skip all symbols}\\
\delta(q_2, B) &= (q_1, B, R) \text{ // and reach the begin}\\
&\text{ // end: while}\\
\delta(q_1, \#) &= (q_3, \#, R) \text{ // skip the \#}\\
\delta(q_3, B) &= (q_f, B, S) \text{// and verify no symbols after \#, accept
input}
\end{align*}

\section{1-turn TM to PDA}
We can construct PDA M' to simulate M by converting each TM transition into a
PDA transition as follows:
\begin{align*}
\gamma_1, \gamma_2, \gamma_3 &\in \Gamma\\
\delta(q_1, \gamma_1) = (q_2, \gamma_2, L) &\to
\delta(q_1, \Lambda, \gamma_1) = (q_2, \Lambda)\\
\delta(q_1, \gamma_1) = (q_2, \gamma_2, R) &\to
\delta(q_1, \gamma_1, \gamma_3) = (q_2, \gamma_2 \gamma_3)\\
\delta(q_1, \gamma_1) = (q_2, \gamma_2, S) &\to
\delta(q_1, \Lambda, \gamma_1) = (q_2, \gamma_2)
\end{align*}

\section{TM to 2-stack PDA}
We can construct 2-stack PDA M' to simulate M by converting each TM transition
into a 2-stack PDA transition as follows:
\begin{align*}
\gamma_1, \gamma_2, \gamma_3 &\in \Gamma\\
\delta(q_1, \gamma_2) = (q_2, \gamma_3, L) &\to
\delta(q_1, \Lambda, \gamma_1, \gamma_2) = (q_2, \Lambda, \gamma_1 \gamma_3)\\
\delta(q_1, \gamma_2) = (q_2, \gamma_3, R) &\to
\delta(q_1, \Lambda, \gamma_1, \gamma_2) = (q_2, \gamma_3 \gamma_1, \Lambda)\\
\delta(q_1, \gamma_2) = (q_2, \gamma_3, S) &\to
\delta(q_1, \Lambda, \gamma_1, \gamma_2) = (q_2, \gamma_1, \gamma_3)
\end{align*}
To initialize M', store input on stack 2 as follows:
\begin{align*}
\delta(q_{-2}, \gamma_3, \gamma_1, Z) &= (q_{-2}, \gamma_3 \gamma_1, Z)\\
\delta(q_{-2}, \Lambda, \gamma_1, Z) &= (q_{-1}, \gamma_1, Z)\\
\delta(q_{-1}, \Lambda, \gamma_1, \gamma_2) &= (q_{-1}, \Lambda, \gamma_1 \gamma_2)\\
\delta(q_{-1}, \Lambda, Z, \gamma_2) &= (q_0, Z, \gamma_2)
\end{align*}
\end{document}