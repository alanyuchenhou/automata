\documentclass{article}
\usepackage{listings}
\usepackage{amsmath}
\usepackage{fullpage}
\usepackage{tabularx}
\usepackage{graphicx}
\usepackage{cite}
\begin{document}
\lstset{language=Java}
\title{CS317 homework 10}
\author{Yuchen Hou}
\maketitle

\section{Turing machine and recursively enumerable language}
\begin{align*}
\gamma &= b, c\\
\delta(q_0, \gamma) &= (q_1, \gamma, R) \text{ // find 1 b or
1 c}\\
\delta(q_1, \gamma) &= (q_1, \gamma, R) \text{ // skip all b and
c}\\
\delta(q_1, \#) &= (q_2, \#, R) \text{ // and find \#}\\
\delta(q_2, a) &= (q_3, a, R) \text{ // find 1 a}\\
\delta(q_3, a) &= (q_3, a, R) \text{ // skip all a}\\
\delta(q_3, B) &= (q_f, B, S) \text{ // and find the end, enter final state,
accept input}
\end{align*}

\section{Turing machine and recursively enumerable language}
\begin{align*}
\gamma &= a, b\\
\gamma_1 &= a, b, \#\\
\delta(q_0, \gamma) &= (q_1, \gamma, S) \text{ // assert the first symbol is a
or b}\\
&\text{ // begin: while (first symbol != \#)}\\
\delta(q_1, \gamma) &= (q_{\gamma,+}, B, R) \text{ // read the first symbol,
remember it, remove it from string}\\
\delta(q_{\gamma,+}, \gamma_1) &= (q_{\gamma,+}, \gamma_1, R) \text{ // skip all
symbols}\\
\delta(q_{\gamma,+}, B) &= (q_{\gamma,-}, B, L) \text{ // and find the last
symbol}\\
\delta(q_{\gamma,-}, \gamma) &= (q_2, B, L) \text{ // assert the last symbol is
the first symbol, remove it from string}\\
\delta(q_2, \gamma_1) &= (q_2, \gamma_1, L) \text{ // skip all symbols}\\
\delta(q_2, B) &= (q_1, B, R) \text{ // and find the first symbol}\\
&\text{ // end: while}\\
\delta(q_1, \#) &= (q_3, \#, R) \text{ // skip the \#}\\
\delta(q_3, B) &= (q_f, B, S) \text{// and assert no leftover symbol after \#,
enter final state}
\end{align*}

\section{PDA simulating one-turn Turing machine}
 A one-turn Turing machine has 3 types of transitions:
\begin{align*}
\gamma_1, \gamma_3 &\in \Gamma\\
\delta(q_1, \gamma_1) &= (q_2, \gamma_3, L) \text{ //type 1}\\
\delta(q_1, \gamma_1) &= (q_2, \gamma_3, R) \text{ //type 2}\\
\delta(q_1, \gamma_1) &= (q_2, \gamma_3, S) \text{ //type 3}
\end{align*}
and 2 stages:
\begin{enumerate}
  \item move to the right (only 2nd and 3rd type of transitions above)
  \item move to the left (only 1st and 3rd type of transitions above)
\end{enumerate}
The PDA needs to simulate stage 1 by reading input, pushing symbols to stack,
modifying the stack top as follows:
\begin{align*}
\gamma_1, \gamma_2, \gamma_3 &\in \Gamma\\
\delta(q_1, \gamma_1, \gamma_2) &= (q_2, \gamma_3 \gamma_2) \text{ //type 2}\\
\delta(q_1, \gamma_1, \gamma_2) &= (q_2, \gamma_3) \text{ //type 3}
\end{align*}
The PDA needs to simulate stage 2 by reading nothing, popping symbols from
stack, modifying the stack top as follows:
\begin{align*}
\gamma_1, \gamma_2, \gamma_3 &\in \Gamma\\
\delta(q_1, \Lambda, \gamma_2) &= (q_2, \Lambda) \text{ //type 1}\\
\delta(q_1, \Lambda, \gamma_2) &= (q_2, \gamma_3) \text{ //type 3}
\end{align*}
Note that type 3 transition is not atomic, but a short hand of 2 consecutive
transitions as follows:
\begin{align*}
\gamma_1, \gamma_2, \gamma_3, \gamma_9 &\in \Gamma\\
\delta(q_1, \gamma_1, \gamma_2) &= (q_{\gamma_3, q_2}, \Lambda)\\
\delta(q_{\gamma_3, q_2}, \Lambda, \gamma_2) &= (q_2, \gamma_3\gamma_2)
\end{align*}
In the initial state, to simulate head pointing to the first symbol of the tape,
the stack is empty.

\section{2-stack PDA simulating Turing machine}
 A Turing machine has 3 types of transitions:
\begin{align*}
\gamma_1, \gamma_3 &\in \Gamma\\
\delta(q_1, \gamma_1) &= (q_2, \gamma_3, L) \text{ //type 1}\\
\delta(q_1, \gamma_1) &= (q_2, \gamma_3, R) \text{ //type 2}\\
\delta(q_1, \gamma_1) &= (q_2, \gamma_3, S) \text{ //type 3}
\end{align*}
For a 2-stack PDA to simulate this, it needs to store the tape in its
2 stacks (left stack and right stack): simulate the part of the tape to the left
of the head(inclusive, with the first symbol at stack bottom, to simulate the
head symbol with the top of left stack) with the left stack, and simulate the
part of the tape to the right of the head with the right stack(exclusive, with
the last symbol at the stack bottom). Then the PDA can simulate 3 types of
Turing machine transitions as follows:
\begin{align*}
\gamma_1, \gamma_2, \gamma_3 &\in \Gamma\\
\delta(q_1, \gamma_1, \gamma_1, \gamma_2) &= (q_2, \Lambda, \gamma_3 \gamma_2) \text{ //type 1}\\
\delta(q_1, \gamma_1, \gamma_1, \gamma_2) &= (q_2, \gamma_3 \gamma_1, \Lambda) \text{ //type 2}\\
\delta(q_1, \gamma_1, \gamma_1, \gamma_2) &= (q_2, \gamma_3, \gamma_2) \text{ //type 3}
\end{align*}
Note that type 3 transition is not atomic, but a short hand of 2 consecutive
transitions as follows:
\begin{align*}
\gamma_1, \gamma_2, \gamma_3, \gamma_9 &\in \Gamma\\
\delta(q_1, \gamma_1, \gamma_1, \gamma_2) &= (q_{\gamma_3, q_2}, \Lambda, \gamma_2)\\
\delta(q_{\gamma_3, q_2}, \Lambda, \gamma_9, \gamma_2) &= (q_2, \gamma_3\gamma_9, \gamma_2)
\end{align*}
In the initial state, to simulate head pointing to the first symbol of the tape,
the left stack is empty, and the right stack if filled with tape symbols.
\end{document}